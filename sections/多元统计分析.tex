\section{多元统计分析部分}\label{SecMultivariateStatisticalAnalysis}
\begin{center}
    Instructor: Dong Li \& Tianying Wang
\end{center}
\subsection{Multivariate Data}
    In this section, we consider a \textbf{Multivariate Statistic Model}. Sample comes from $p$ dimension multivariate population $f(x_1,x_2,\ldots,x_p)$.

    \textbf{Notation }: In this section, we still denote random variable in upper case and observed value in lower case, specially express random vector in bold font. \textbf{But} in this section we usually omit the vector symbol $ \vec{\cdot} \,\,$. e.g.
    random vector with $ n $ \textbf{variable }is denoted as $\mathbf{X}=(X_{\cdot 1},X_{\cdot 2},\ldots ,X_{\cdot p})$; sample of size $ n $ from the multivariate population is a $ n\times p $ matrix $ \{x_{ij}\} $, each sample item (a row in sample matrix) is denoted as $ x_i' $ or $ x_i^T $.\footnote{Here sample item (or sample case) $x_i=[x_{i1},x_{i2},\ldots,x_{ip}]^T$ is a column vector.} 
    % In this section we use the upper case $ X_i $ means that it's a vector (not necessarily means an r.v.).
    %\footnote{In previous section, a multivariate r.v. is denoted $\vec{X}=(X_1,X_2,\ldots,X_p) $, and sample item is $ \vec{X_i}=(X_{i1},X_{i2},\ldots,X_{ip})  $}


\subsubsection{Matrix Representation}


    \begin{itemize}[topsep=0pt,itemsep=1pt]
        \item \hyperlink{RandomVariableRepresentation}{Random Variable Representation}
        \item \hyperlink{SampleRepresentation}{Sample Representation}
        \item \hyperlink{StatisticsRepresentation}{Statistics Representation}
        \item \hyperlink{SampleStatisticsProperties}{Sample Statistics Properties}
    \end{itemize}
    



\begin{point}
    \hypertarget{RandomVariableRepresentation}{Random Variable Representation}:
\end{point}
    \begin{itemize}[topsep=6pt,itemsep=4pt]
        \item Random Matrix: Definition and basic properties of r.v. see section \ref{SectionPropertiesOfRandomVariableAndVector}. Now extend the definition to matrix $ X=\{X_{ij}\} $. 
    
    \begin{equation}
        X=\{X_{ij}\}=\begin{bmatrix}
        X_{11}&X_{12}&\ldots&X_{1p}\\
        X_{21}&X_{22}&\ldots&X_{2p}\\
        \vdots&\vdots&\ddots&\vdots\\
        X_{1n}&X_{n2}&\ldots&X_{np}\\
        \end{bmatrix} 
    \end{equation}

    And we can further define $ E(X)=\{E(X_{ij})\} $.
    For any const matrix $ A,B $ we have
    \begin{equation}
        E(AXB)=AE(X)B 
    \end{equation}

    \item Random Vector: For a $ p\times 1 $ random vector $ \vec{X}=(X_{1},X_{2},\ldots,X_{p})^T  $, denote (Marginal) expectation and variance, and covariance, correlation coefficient between $ X_i,X_j $ as follows:
    \begin{align*}
        \mu_i&=E(X_i)\\
        \sigma _{ii}&=\sigma_i ^2=E(X_i-\mu_i)^2\\
        \sigma_{ij}&=E[(X_i-\mu_i)(X_j-\mu_j)]\\
        \rho _{ij}&=\dfrac{\sigma _{ij}}{\sqrt{\sigma _{ii}}\sqrt{\sigma _{jj}}}
    \end{align*}
    
    and we have covariance matrix (as defined in section \ref{SubSubSectionCovarianceAndCorrelation}, eqa.\ref{covariancematrix})
    \begin{equation}
        \Sigma =E[(X-\mu)(X-\mu)^T] =
        \begin{bmatrix}
        \sigma _{11}&\sigma _{12}&\ldots&\sigma _{1p}\\
        \sigma _{21}&\sigma _{22}&\ldots&\sigma _{2p}\\
        \vdots&\vdots&\ddots&\vdots\\
        \sigma _{1p}&\sigma _{p2}&\ldots&\sigma _{pp}\\
        \end{bmatrix}
    \end{equation}

    and Standard Deviation Matrix
    \begin{equation}
        V^{1/2}=diag\{\sqrt{\sigma _{ii}}\} 
    \end{equation}

    Based on $ \vec{X}=(X_{1},X_{2},\ldots,X_{p})  $, consider the linear combination:$ Y=c'X=c_1X_1+c_2X_2+\ldots c_pX_p $
    \begin{align*}
        E(y)=c'\mu\qquad var(Y)=c'\Sigma c
    \end{align*}

    and $ Z_i=\sum_{j=1}^p c_{ij}X_j $ (i.e. $ Z=CX $):
    \begin{equation}
        \mu_Z=E(Z)= C\mu_X\qquad \Sigma _Z=C\Sigma _XC^T
    \end{equation}
    
    
    
    

    and Correlation Matrix
    \begin{equation}
        \rho =\begin{bmatrix}
        \rho _{11}&\rho _{12}&\ldots&\rho _{1p}\\
        \rho _{21}&\rho _{22}&\ldots&\rho _{2p}\\
        \vdots&\vdots&\ddots&\vdots\\
        \rho _{1p}&\rho _{p2}&\ldots&\rho _{pp}\\
        \end{bmatrix} 
        =V^{-1/2}\Sigma V^{-1/2}
    \end{equation}
    
    

    
    
    
    
    
    \end{itemize}
    
        
\begin{point}
    \hypertarget{SampleRepresentation}{Sample Representation}:
\end{point}
    
    Sample of $n$ items from population characterized by $ p $ variables
    \begin{table}[H]
        \centering
        \begin{tabular}{|c|cccccc|}
            \hline
            \diagbox{Item}{Variable}&Variable 1&Variable 2&$\ldots$&Variable $j$&$\ldots$&Variable $p$\\
            \hline
            Item 1&$ x_{11} $&$ x_{12} $&$ \ldots $&$ x_{1j} $&$ \ldots $&$ x_{1p} $\\
            Item 1&$ x_{21} $&$ x_{22} $&$ \ldots $&$ x_{2j} $&$ \ldots $&$ x_{2p} $\\
            $\vdots$&$\vdots$&$\vdots$&$ \ddots $&$\vdots$&$ \ddots $&$\vdots$\\
            Item $j$&$ x_{i1} $&$ x_{i2} $&$ \ldots $&$ x_{ij} $&$ \ldots $&$ x_{ip} $\\
            $\vdots$&$\vdots$&$\vdots$&$ \ddots $&$\vdots$&$ \ddots $&$\vdots$\\            
            Item $n$&$ x_{n1} $&$ x_{n2} $&$ \ldots $&$ x_{nj} $&$ \ldots $&$ x_{np} $\\
            \hline
        \end{tabular}
    \end{table}

    Or represented in condense notation:
    \begin{equation}
        X=\{x_{ij}\}=
        \begin{bmatrix}
            x_1^T\\x_2^T\\ \vdots \\ x_n^T
        \end{bmatrix}
        =
        \begin{bmatrix}
            x_{11}&x_{12}&\ldots&x_{1p}\\
            x_{21}&x_{22}&\ldots&x_{2p}\\
            \vdots&\vdots&\ddots&\vdots\\
            x_{n1}&x_{n2}&\ldots&x_{np}\\
        \end{bmatrix} 
        =
        \begin{bmatrix}
            y_1&y_2&\ldots &y_p
        \end{bmatrix}
    \end{equation}
\begin{point}
    \hypertarget{StatisticsRepresentation}{Statistics Representation}
\end{point}

    \begin{itemize}[topsep=6pt,itemsep=4pt]
        \item Unit 1 vector:
        \begin{equation}
            \mathbf{1}_k=(\underbrace{1,1,\ldots,1}_{k\text{ 1 in total}})^T
        \end{equation}

        Unit 1 matrix:
        \begin{equation}\label{EqaAllOneMatrix}
            \mathcal{J}_n  = \{1\}_{n\times n}=\begin{bmatrix}
            1&1&\ldots&1\\
            1&1&\ldots&1\\
            \vdots&\vdots&\ddots&\vdots\\
            1&1&\ldots&1\\
            \end{bmatrix}_{n\times n}
        \end{equation}
        

        \item Sample mean:
        \begin{equation}
            \bar{x}_i=\dfrac{x_{1i}+x_{2i}+\ldots+x_{ni}}{n}=\dfrac{y_i'\mathbf{1}_n}{n}
        \end{equation}
        
        \item Deviation of measurement of the $ i^\mathrm{th} $ variable:
        \begin{equation}
            d_i=y_i-\bar{x}_i\mathbf{1}_n=\begin{bmatrix}
                x_{1i}-\bar{x}_i\\x_{2i}-\bar{x}_i\\\vdots\\x_{ni}-\bar{x}_i
            \end{bmatrix} 
        \end{equation}
        \item Covariance Matrix:
            \begin{itemize}[topsep=6pt,itemsep=4pt]      
            \item Variance of $ y_i $:
            \begin{equation}
                s^2_i=s_{ii}=\dfrac{1}{n}d_i'd_i =\dfrac{1}{n}\sum_{k=1}^n (x_{ki}-\bar{x}_i)^2,\quad i=1,2,\ldots p
            \end{equation}
            \item Covariance between $ y_i $ and $ y_j $:
            \begin{equation}
                s_{ij}=\dfrac{1}{n}d_i'd_j=\dfrac{1}{n}\sum_{k=1}^n(x_{ki}-\bar{x}_i)(x_{kj}-\bar{x}_j),\quad i,j=1,2,\ldots p
            \end{equation}
            \item Correlation Coefficient:
            \begin{equation}\label{EqaEstimatorOfCorrelationCoefficient}
                r_{ij}=\dfrac{s_{ij}}{\sqrt{s_{ii}}\sqrt{s_{jj}}}=\dfrac{{\displaystyle\sum_{k=1}^n(x_{ki}-\bar{x}_i)(x_{kj}-\bar{x}_j)}}{\sqrt{{\displaystyle\sum_{k=1}^n(x_{ki}-\bar{x}_i)^2}}\sqrt{{\displaystyle\sum_{k=1}^n(x_{kj}-\bar{x}_j)^2}}},\quad i,j=1,2,\ldots p
            \end{equation}
            \end{itemize}
        
        In condense notation, define Covariance Matrix from sample of size $ n $:
        \begin{equation}
            S^2_n=\begin{bmatrix}
            s_{11}&s_{12}&\ldots&s_{1p}\\
            s_{21}&s_{22}&\ldots&s_{2p}\\
            \vdots&\vdots&\ddots&\vdots\\
            s_{1p}&s_{p2}&\ldots&s_{pp}\\
            \end{bmatrix}
        \end{equation}

        and sample Correlation Coefficient Matrix:
        \begin{equation}
            R_n=
            \begin{bmatrix}
            r_{11}&r_{12}&\ldots&r_{1p}\\
            r_{21}&r_{22}&\ldots&r_{2p}\\
            \vdots&\vdots&\ddots&\vdots\\
            r_{1p}&r_{p2}&\ldots&r_{pp}\\
            \end{bmatrix}
        \end{equation}
        \item Generalized sample variance: $ |S|=\lambda _1\lambda _2 \ldots \lambda _p$, where $ \lambda_i  $ are eigenvalues.
        
        \item 'Statistical Distance' between vectors: to measure the difference between two vectors $ x=(x_1,x_2,\ldots,x_p) $ and $ y=(y_1,y_2,\ldots,y_p) $.
        \begin{itemize}[topsep=6pt,itemsep=4pt]
            \item Euclidean Distance:
            \begin{equation}
                d_E(x,y) =\sqrt{(x-y)^T(x-y)}
            \end{equation}
            \item \textbf{Mahalanobis Distance}\index{Mahalanobis Distance}: Scale invariant distance, and include information about relativity:
            \begin{equation}
                d_M(x,y)=\sqrt{(x-y)'S^{-1}(x-y)} 
            \end{equation}

            Note: $ P,Q $ are from the same distribution with covariance matrix $ S_p $. When $ S=I $, return to Euclidean distance.
            
            Remark: Mahalanobis distance is actually the normalized Euclidean distance in principal component space. So we can actually define the Mahalanobis distance for one sample case $ \vec{x}=(x_1,x_2,\ldots ,x_p) $ from distribution of $ (\vec{\mu},\Sigma)  $
            \begin{equation}\label{MahalanobisDistance}
                d_M(\vec{x})=\sqrt{(\vec{x}-\vec{\mu})^T\Sigma ^{-1}(\vec{x}-\vec{\mu})} 
            \end{equation}

            Note: the hyper-sruface $ d_M(\vec{x}) $ forms a ellipsoid.

        \end{itemize}
    \end{itemize}

\begin{point}
    \hypertarget{SampleStatisticsProperties}{Sample Statistics Properties}
\end{point}

    Consider take an $ n $ cases sample from r.v. population $ \vec{X}=(X_1,X_2,\ldots,X_p) $, population mean $ \mu $ and covariance matrix $ \Sigma  $.
    \begin{itemize}[topsep=6pt,itemsep=4pt]
        \item $ E(\bar{\bar{X}})=\mu $;
        \item $ cov(\bar{X})=\dfrac{1}{n}\Sigma  $;
        \item $ E(S_n)=\dfrac{n-1}{n}\Sigma  $
    \end{itemize}
    
        


\subsubsection{Review: Some Matrix Notation \& Lemma}\label{SubSubSectionMatrixNotationAndLemma}

    \begin{itemize}[topsep=6pt,itemsep=4pt]
        \item Orthonormality: For square matrix $ P $ satisfies:
        \begin{equation}
            x_i^Tx_j=\delta _{ij} 
        \end{equation}

        where $ x_i,x_j $ are columns of $ P $.
        \item Eigenvalue and Eigenvector: For square matrix $ A $, its eigenvalues $ \lambda_i $ and corresponding eigenvectors $ e_i $ satisfies:
        \begin{equation}
            Ae_i=\lambda_ie_i,\,\forall i=1,2,\ldots p 
        \end{equation}

        Denote $ P=[e_1,e_2,\ldots ,e_p] $, which is an orthonormal matrix. And denote $ \Lambda =diag\{\lambda _1,\lambda _2,\ldots,\lambda _p\} $.
        \begin{equation}
            A=\sum_{i=1}^p\lambda _ie_ie_i^T=P \Lambda P^T=P\Lambda P^{-1}
        \end{equation}

        is called the Spectral Decomposition of $ A $

        
        
        \item Square root matrix: Def. as
        \begin{equation}
            A^{1/2}=\sum_{i=1}^p\sqrt{\lambda _i}e_ie_i^T=P\Lambda ^{1/2}P^T 
        \end{equation}

        Properties:
        \begin{itemize}[topsep=0pt,itemsep=-2pt]
            \item $ {\displaystyle A^{1/2}A^{1/2}A} $;
            \item $ {\displaystyle A^{-1/2}=(A^{1/2})^{-1}=PL^{-1/2}}P^T $;
            \item $ tr(A) =\sum_{i=1}^n\lambda _n$;
            \item $ |A|=\prod_{i=1}^n\lambda _n $.
        \end{itemize}
        
            
        \item (Symmetric) Positive Definite Matrix: Say $ A $ a Positive Definite Matrix if
        \begin{equation}
            x^TAx> 0,\,\forall x\in\mathbb{R}^p 
        \end{equation}

        where $ x^TAx $ is called a Quadric Form.

        Properties:
        \begin{itemize}[topsep=6pt,itemsep=4pt]
            \item Use the Spectral Decomposition of $ A $, we can write the Quadric Form as
            \begin{equation}
                x^TAx=x^TP\Lambda P^Tx=y^T\Lambda y=\sum_{i=1}^p\lambda_iy_i^2=\sum_{i=1}^p(\sqrt{\lambda_i}y_i)^2 
            \end{equation}
            
            
            \item Eigenvalues $ \lambda _i>0,\,\forall i=1,2,\ldots,p $
            \item $ A $ can be written as product of symmetric matrix: $ A= Q^TQ$ ($ Q $ is symmetric);
        \end{itemize}

        \item Trace of Matrix: For $ p\times p $ square matrix $ A $
            
            \begin{equation}
                tr(A) =\sum_{i=1}^p a_{ii}
            \end{equation}
            
            Properties:
            \begin{itemize}[topsep=2pt,itemsep=2pt]
                \item $ tr(AB)=tr(BA)  $;
                \item $ x'Ax=tr(x'Ax)=tr(Axx') $
            \end{itemize}
            
                
        \item Matrix Partition: partition matrix $ \mathop{A}\limits_{p\times p} $ as 
        \[
            A=         
            \begin{bmatrix}
                \mathop{A_{11} }\limits_{q_1\times q_1}&\mathop{A_{12} }\limits_{q_1\times q_2} \\
                \mathop{A_{21} }\limits_{q_2\times q_1}&\mathop{A_{22} }\limits_{q_2\times q_2}   
            \end{bmatrix}   
        \]

        where $ p=q_1+q_2 $
        
        Property:
        \[
            |A|= |A_{22}||A_{11}-A_{12}A_{22}^{-1}A_{21}|=|A_{11}||A_{22}-A_{21}A_{11}^{-1}A_{12}|
        \]
        
        
        | A |=| A22 || A11 - A12A-122A21 |=| A11 || A22 - A21A-111A12 |
                
            
                       
        \item Calculus Notations: We want to take derivative of $ y=(y_1,y_2,\ldots,y_q)^T $ over $ x=(x_1,x_2,\ldots,x_p)^T $
        
        We use 'Denominator-layout', which is
        \begin{equation}
            \dfrac{\partial^{}y }{\partial ^{}x}=\dfrac{\partial^{} y^T}{\partial x^{}} =
            \begin{bmatrix}
            \dfrac{\partial^{} y_1}{\partial x_1 ^{}}&\dfrac{\partial^{} y_2}{\partial x_1 ^{}}&\ldots&\dfrac{\partial^{} y_q}{\partial x_1 ^{}}\\
            \dfrac{\partial^{} y_1}{\partial x_2 ^{}}&\dfrac{\partial^{} y_2}{\partial x_2 ^{}}&\ldots&\dfrac{\partial^{} y_2}{\partial x_p ^{}}\\
            \vdots&\vdots&\ddots&\vdots\\
            \dfrac{\partial^{} y_1}{\partial x_p ^{}}&\dfrac{\partial^{} y_2}{\partial x_p ^{}}&\ldots&\dfrac{\partial^{} y_q}{\partial x_p ^{}}\\
            \end{bmatrix}
        \end{equation}
        
        \hypertarget{MatrixDifferenciation}{Properties (under denominator-layout):}
        \begin{itemize}[topsep=6pt,itemsep=2.5pt]
            \item $ \dfrac{\partial^{} }{\partial x^{}}Ax=A^T $;\\
            \item $ \dfrac{\partial^{} }{\partial x^{}}x^TA=A $;\\
            \item $ \dfrac{\partial^{} }{\partial x^{}}x^Tx=2x $;\\
            \item $ \dfrac{\partial^{} }{\partial x^{}}x^TAx=Ax+A^Tx $;\\
            \item $ \dfrac{\partial^{} }{\partial x^{}}\log(x^TAx)=\dfrac{2Ax}{x^TAx} $;\\
            \item $ \dfrac{\partial^{} |A|}{\partial A^{}}=|A|A^{-1} $;\\
            \item $ \dfrac{\partial^{} tr(AB)}{\partial A^{}}=B^T $;\\
            \item $ \dfrac{\partial^{} tr(A^{-1}B)}{\partial A^{}}=-A^{-1}B^TA^{-1} $
        \end{itemize}
          
        
        \item Kronecker Product: For matrix $ \mathop{A}\limits_{m\times n}=\{a_{ij}\},\,\mathop{B}\limits_{p\times q}=\{b_{ij}\} $. Their Kronecker product
        \begin{equation}
            A\otimes B=\begin{bmatrix}
            a_{11}B&a_{12}B&\ldots&a_{1n}B \\
            a_{21}B&a_{22}B&\ldots&a_{2n}B \\
            \vdots&\vdots&\ddots&\vdots\\
            a_{1m}B&a_{m2}B&\ldots&a_{mn}B \\
            \end{bmatrix} 
        \end{equation}
        
    \end{itemize}
    
        


    \subsubsection{Useful Inequalities}
    \begin{itemize}[topsep=6pt,itemsep=4pt]
        \item Cauchy-Schwartz Inequality:\index{Inequality!Cauchy-Schwarz Inequality}
        
        Let $ b,d$ are any $ p\times 1 $ vectors.
        \begin{equation}
            (b'd)^2\leq (b'b)(d'd) 
        \end{equation}
        
        \item Extended Cauchy-Schwartz Inequality: 
        
        Let $ B $ be a positive definite matrix.
        
        \begin{equation}
            (b'd)^2\leq(b'Bb)(d'B^{-1}d) 
        \end{equation}
        
        \item Maximazation Lemma:\index{Inequality!Maximazation Lemma}
        
        $ d $ be a given vector, for any non-zero vector $ x $,
        \begin{equation}
            \dfrac{(x'd)^2}{x'Bx}\leq d'B^{-1}d 
        \end{equation}

        Take Maximum when $ x=cB^{-1}d $.
        
        
    \end{itemize}

    % note: 无法用地位投影寻找高微离群值
        

\subsection{Statistical Inference to Multivariate Population}
    Statistics model: a $ n $ cases sample $ \mathbf{X}_1,\mathbf{X}_2,\ldots,\mathbf{X}_n $, where each $ \mathbf{X}_i $ i.i.d. from a multivariate population (usually consider a multi-normal). i.e.
    \begin{equation}\label{EqaNPSampleMatrixNotation}
        \mathbf{X}=\begin{bmatrix}
            X_{11}&X_{12}&\ldots&X_{1p}\\
            X_{21}&X_{22}&\ldots&X_{2p}\\
            \vdots&\vdots&\ddots&\vdots\\
            X_{1n}&X_{n2}&\ldots&X_{np}\\
            \end{bmatrix} 
            =
            \begin{bmatrix}
                \mathbf{X}_1'\\
                \mathbf{X}_2'\\
                \vdots\\
                \mathbf{X}_n'
            \end{bmatrix}
    \end{equation}



\subsection{Multivariate Normal Distribution}
    Univariate Noraml Distribution: $ N(\mu,\sigma^2) $
    \begin{equation}
        f_X(x)=\dfrac{1}{\sqrt{2\pi\sigma ^2}}\exp{-\dfrac{(x-\mu)^2}{2\sigma ^2}} 
    \end{equation}
    
    Multivariate Normal Distribution: $X\sim N_p(\vec{\mu},\Sigma) $\footnote{Detailed derivation see section \ref{SubsectionDerivationMultivariateNormal}}
    \begin{equation}
        f_\mathbf{X}(\vec{x})=\dfrac{1}{(2\pi)^{p/2}|\Sigma |^{1/2}}\exp\left({-\dfrac{(\vec{x}-\vec{\mu})'\Sigma^{-1}(\vec{x}-\vec{\mu})}{2}} \right)
    \end{equation}

    Note: Here in the $ \exp $, the $ (\vec{x}-\vec{\mu})'\Sigma^{-1}(\vec{x}-\vec{\mu}) $ is the Mahalanobis Distance $ d_M $ defined in eqa.\ref{MahalanobisDistance}

    % Further denote $ \mathop{Y}\limits_{q\times 1}=\mathop{A}\limits_{q\times p}\mathop{X}\limits_{p\times 1} $, where $ A $ is a const matrix. Then 
    % \begin{equation}
    %     Y=AX\sim N_q(A\vec{\mu},A\Sigma A^T) 
    % \end{equation}
    
    

    Remark: A $ n $-dimension multivariate normal has $ \dfrac{p(p+1)}{2} $ free parameters. Thus for a very high dimension, contains too many free parameters to be determined! 
    
    Properties: Consider $ X\sim N_p(\mu,\Sigma) $
    \begin{itemize}[topsep=6pt,itemsep=4pt]
        \item Linear Transform:
        \begin{itemize}[topsep=6pt,itemsep=4pt]       
        \item For a $ p\times 1 $ vector $ a $:
        \begin{equation}
            X\sim N_p(\mu,\Sigma )\Leftrightarrow a'X\sim N(a'\mu,a'\Sigma a),\,\forall a\in\mathbb{R}^p 
        \end{equation}

        (Proof: use characteristic function.)
        
        \item For a $ q\times p $ const matrix $ A $:
        \begin{equation}\label{EqaTransformOfMultiNormal}
            AX+a\sim N_q(A\mu+a,A\Sigma  A')
        \end{equation}
        \item For a $ p\times p    $ square matrix $ A $:
        
        \begin{equation}\label{EqaExpectationOfQuadric}
            E(X'AX)= \mu'A\mu +tr(A\Sigma )            
        \end{equation}
        
        
        \end{itemize}
        \item Conditional Distribution: Take partition of $ \mathop{X}\limits_{p\times 1}\sim N(\mathop{\mu}\limits_{p\times 1},\mathop{\Sigma }\limits_{p\times p}) $ into $ \mathop{X_1}\limits_{q_1\times 1} $ and $ \mathop{X_2}\limits_{q_2\times 1}  $, where $ q_1+q_2=p $. Write in matrix form:
        \begin{equation}
            \mathop{X}\limits_{p\times 1}=
            \begin{bmatrix}
                \mathop{X_1}\limits_{q_1\times 1}\\
                \mathop{X_2}\limits_{q_2\times 2}  
            \end{bmatrix}  
            \qquad 
            \mathop{\mu}\limits_{p\times 1}=
            \begin{bmatrix}
                \mathop{\mu_1 }\limits_{q_1\times 1}\\
                \mathop{\mu_2 }\limits_{q_2\times 2}  
            \end{bmatrix}  
            \qquad             
            \mathop{\Sigma }\limits_{p\times p}=
            \begin{bmatrix}
                \mathop{\Sigma_{11} }\limits_{q_1\times q_1}&\mathop{\Sigma_{12} }\limits_{q_1\times q_2} \\
                \mathop{\Sigma_{21} }\limits_{q_2\times q_1}&\mathop{\Sigma_{22} }\limits_{q_2\times q_2}   
            \end{bmatrix}  
            \qquad 
        \end{equation}
        
            i.e. 
        \begin{equation}
            \mathop{X}\limits_{p\times 1}=\begin{bmatrix}
                \mathop{X_1 }\limits_{q_1\times 1}\\
                \mathop{X_2 }\limits_{q_2\times 2}  
            \end{bmatrix}  
            \sim
            N_{q_1+q_2}\left(\begin{bmatrix}
                \mathop{\mu_1 }\limits_{q_1\times 1}\\
                \mathop{\mu_2 }\limits_{q_2\times 2}  
            \end{bmatrix},\begin{bmatrix}
                \mathop{\Sigma_{11} }\limits_{q_1\times q_1}&\mathop{\Sigma_{12} }\limits_{q_1\times q_2} \\
                \mathop{\Sigma_{21} }\limits_{q_2\times q_1}&\mathop{\Sigma_{22} }\limits_{q_2\times q_2}   
            \end{bmatrix}  
                \right)
        \end{equation}
            
        Independence: $ X_1\parallel X_2\Leftrightarrow \Sigma _{21}=\Sigma _{12}^T=0  $

        And the conditional dictribution $ X_1|X_2=x_2 $ is given by \footnote{In eqa(\ref{EqaTransformOfMultiNormal}), take 
        \[
            \mathop{A}\limits_{p\times p}=\begin{bmatrix}
                \mathop{I}\limits_{q\times q} & -\mathop{\Sigma _{12}\Sigma _{22}^{-1}}\limits_{q\times (p-q)} \\
                \mathop{0}\limits_{(p-q)\times q}&\mathop{I}\limits_{(p-q)\times (p-q)}  
            \end{bmatrix}  
        \]
        
        }
        \begin{equation}
            X_1|_{X_2=x_2}\sim N_p(\mu_1+\Sigma _{12}\Sigma _{22}^{-1}(x_2-\mu_2),\,\Sigma _{11}-\Sigma _{12}\Sigma _{22}^{-1}\Sigma _{21})
        \end{equation}

        \item Multivariate Normal \& $ \chi^2 $
        
         Let $ X\sim N_p(\mu,\Sigma ) $, then 
         \begin{equation}
             (X-\mu)^T\Sigma ^{-1}(X-\mu)\sim \chi_p^2 
         \end{equation}
         
         \item Linear Combination:
        Let $ X_1,X_2\ldots,X_n $ with $ X_i\sim N_p(\mu_i,\Sigma ) $ (different $ \mu_i $, same $ \Sigma  $). And denote $ V_1=\sum_{i=1}^nc_iX_i $, then
        \begin{equation}
            V_1\sim N_p(\sum_{i=1}^n c_i\mu_i,\sum_{i=1}^nc_i^2\Sigma ) 
        \end{equation}
        
        
        
        
        
        
    \end{itemize}
    
        






    % \begin{point}
    %     Problem: Property of 2-D Normal:
    %     \begin{equation}
    %         corr(X,Y)=\rho\Rightarrow corr(X^2,Y^2)=\rho ^2 
    %     \end{equation}
    % \end{point}

    
    
\subsubsection{MLE of Multivariate Normal}
    Under the notation in eqa(\ref{EqaNPSampleMatrixNotation}), i.e. each sample case $ \mathbf{X}_i$ i.i.d. $\sim N_p(\mu,\Sigma ) $, we can get the joint PDF of $ \mathbf{X} $:
    \begin{equation}
        f_{\mathbf{X_1},\ldots,\mathbf{X_n};\mu,\Sigma }(x_1,\ldots,x_n)=\dfrac{1}{(2\pi)^{np/2}|\Sigma |^{n/2}}\exp\left( -\sum_{i=1}^n\dfrac{(x_i-\mu)'\Sigma ^{-1}(x_i-\mu)}{2} \right) 
    \end{equation}
  
    and at the same time get likelihood function\footnote{Here we need to use the property of trace
    \begin{equation}
        x'Ax=tr(x'Ax)=tr(Ax'x)
    \end{equation}    }:
    
    \begin{equation}
        L(\mu ,\Sigma;x_1,\ldots,x_n)=\dfrac{1}{(2\pi)^{np/2}|\Sigma |^{n/2}}\exp\left[ -\dfrac{1}{2}tr\left( \Sigma ^{-1} \left(\sum_{i=1}^n(x_i-\bar{x})(x_i-\bar{x})'+n(\bar{x}-\mu)(\bar{x}-\mu)' \right) \right) \right]
    \end{equation}
        And we can get the MLE of $ \mu $ and $ \Sigma  $ as follows\footnote{Detailed proof see '\textit{Applied Multivariate Statistical Analysis}' P130}:
        \begin{align*}
            \hat{\mu}&= \dfrac{1}{n}\sum_{i=1}^n x_i=\bar{x} \\
            \hat{\Sigma  }&= \dfrac{1}{n}\sum_{i=1}^n(x_i-\bar{x})(x_i-\bar{x})'=\dfrac{n-1}{n}S^2
        \end{align*}

    
    And we can furthur construct MLE of function of $ \mu,\,\Sigma  $ (use invariance property of MLE), for example 
    
    \[
        \hat{|\Sigma |}=|\hat{\Sigma }| 
    \]
    
    
        Note: $ (\hat{\mu} , \hat{\Sigma} ) $ is sufficient statistic of multi-normal population.






%Consistency

    % Consistency: Ensuring that when we get more data point, weare 'closer' to the real case.
    % \begin{itemize}[topsep=2pt,itemsep=2pt]
    %     \item Weak consistency:
    %     \begin{equation}
    %         \lim_{n\to\infty}P(||\hat{\mu}-\mu||>\varepsilon )=0 
    %     \end{equation}
    %     \item Strong consistency:
    %     \begin{equation}
    %         \hat{\mu}\xrightarrow[]{\mathrm{a.s.}} \mu 
    %     \end{equation}
    % \end{itemize}
    
        
\subsubsection{Sampling distribution of $ \bar{X} $ and $ S^2 $}
        $ \hat{\mu}=\bar{X} $ and $ \hat{\Sigma}=\dfrac{n-1}{n}S^2 $ are statistics, with sampling distribution.



    \begin{point}
        Sampling distribution of $ \bar{X} $
    \end{point}

    Similar to monovariate case:
    \[
        \bar{X}\sim N_p(\mu,\dfrac{1}{n}\Sigma ) 
    \]
    
    \begin{point}
        Sampling distribution of $ S^2 $
    \end{point}
    
\begin{itemize}[topsep=2pt,itemsep=2pt]
    \item Monovariate case: Consider $ (X_1,X_2,\ldots,X_n) $ i.i.d. $ \sim N(\mu,\sigma ^2) $

    % Then $ \bar{x}=\dfrac{1}{n}\sum_{i=1}^nx_i $, $ S^2=\dfrac{1}{n-1}\sum_{i=1}^n(x_i-\bar{x})^2 $
    
    % Define an orthogonal matrix
    % \begin{equation}
    %     Q=\begin{bmatrix}
    %         \dfrac{1}{\sqrt{n}}&\dfrac{1}{\sqrt{n}}&\ldots&\dfrac{1}{\sqrt{n}}\\
    %         &&&\\
    %         &&&\\
    %         &&&
    %     \end{bmatrix} _{n\times n}
    % \end{equation}
    
    % and def 
    % \begin{equation}
    %     Y=QX\sim N(Q\mathbf{1}_n\mu,\sigma^2I) =N(\begin{bmatrix}
    %         \sqrt{n}\mu\\0\\ \vdots\\0
    %     \end{bmatrix})
    % \end{equation}

    Then 
    \[
        \dfrac{(n-1)S^2}{\sigma ^2}\sim \chi^2_{n-1} 
    \]
    
    \item Multivariate case: Consider $ (\mathbf{X}_{1},\mathbf{X}_{2},\ldots,\mathbf{X}_{n})  $ i.i.d. $ \sim N_p(\mu,\Sigma ) $
    
    Then
    \[
        (n-1) S^2\sim W_p(n-1,\Sigma )
    \]
    
    Where $ W_p(n-1,\Sigma ) $ is Wishart Distribution, details as follows:

         For r.v. $ Z_1,Z_2,\ldots,Z_m $ i.i.d. $ \sim N_p(0,\Sigma  ) $, def $ p $ dimensional \textbf{Wishart Distribution } with dof $ m $ as $ W_p(m,\Sigma ) $.\footnote{$ W_p(m,\Sigma ) $ is a distribution defined on $ p\times p $ matrix space.}
        \begin{equation}
            W_p=\sum_{i=1}^nZ_iZ_i' 
        \end{equation}

        
        PDF of $ W_p(m,\Sigma ) $:
        \begin{equation}
            f_W(w;p,m,\Sigma )= \dfrac{|w|^{\frac{m-p-1}{2}}\exp\left( -\dfrac{1}{2}tr(\Sigma ^{-1}w) \right)}{2^{\frac{mp}{2}}|\Sigma |^{-1/2}\pi^{\frac{p(p-1)}{4}}{\displaystyle\prod_{i=1}^p\Gamma (\dfrac{m-i+1}{2})} }
        \end{equation}

 
        
        C.F.
        \begin{equation}
            \phi(T)=|I_p-2i\Sigma T|^{-\frac{m}{2}} 
        \end{equation}
    % \begin{align*}
    %     \sum_{i=1}^nY_iY_i'=\sum_{i=1^n}X_iX_i'=\sum_{i=1}^n(X_i-\bar{X})(X_i-\bar{X})'+n\bar{X}\bar{X}'=(n-1)S+Y_1Y_1' \\
    %     \Rightarrow (n-1)S=\sum_{i=2}^nY_iY_i'\parallel \bar{X}=\dfrac{1}{\sqrt{n}}Y_1
    % \end{align*} 

    % Then consider the distribution of $ {\displaystyle\sum_{i=2}^nY_iY_i'} \sum W_p(n-1,\Sigma )$, which is Wishart distribution.

    Properties:
    \begin{itemize}[topsep=2pt,itemsep=2pt]
        \item For independent $ A_1\sim W_p(m_1,\Sigma ) $ and $ A_2\sim W_p(m_2,\Sigma ) $, then 
        \[
            A_{1}+A_{2} \sim W_p(m_1+m_2,\Sigma )
        \]
        
        \item For $ A\sim W_p(m,\Sigma ) $, then
        \[
            CAC'\sim W_p(m,C\Sigma C') 
        \]
        \item Wishart distribution is the matrix generization of $ \chi^2_n $. When $ p=1 $, $ \Sigma =\sigma ^2=1 $, $ W_p(m,\Sigma ) $ naturally reduce to $ \chi^2_m $.
        \[
            \chi^2_n=W_1(n,1) 
        \]
        

\begin{rcode}
    Distribution functions are in package \lstinline|MCMCpack|, or use \lstinline|rWishart()| function.
\end{rcode}
        

    
    
\end{itemize}
    
\end{itemize}

    

\begin{point}
    Large sample $ \bar{X} $ and $ S^2 $
\end{point}
\begin{itemize}[topsep=2pt,itemsep=2pt]
    \item $ \sqrt{n}(\bar{X}-\mu)\xrightarrow[]{\mathscr{L}} N_p(0,\Sigma ) $;
    \item $ n(\bar{X}-\mu)'S ^{-1}(\bar{X}-\mu)\xrightarrow[]{\mathscr{L}} \chi_p^2 $
\end{itemize}

    


% Stein's method


\subsection{Multivariate Statistical Inference}\label{SubSectionMultivariateHypothesisTesting}



\subsubsection{Hypothesis Testing for Normal Population}
\begin{itemize}[topsep=2pt,itemsep=2pt]
    \item \textbf{One-Population Hypothesis Testing}: 
    
    Conduct hypothesis testing to $ \mu $:
    \[
        H_0: \mu=\mu_0\longleftrightarrow H_1:\mu\neq \mu_0
    \]

\begin{point}
    Hotelling's $ T^2 $ test
\end{point}

    
    \begin{itemize}[topsep=2pt,itemsep=2pt]
        \item One-Dimensional case: $ t $-test
        \[
            T=\dfrac{\sqrt{n}(\bar{X}-\mu_0)}{S}\sim t_{n-1}
        \]
        
        i.e.
        \[
            T^2=[\sqrt{n}(\bar{X}-\mu_0)](S^2)^{-1}[\sqrt{n}(\bar{X}-\mu_0)] \sim t^2_{n-1}=F_{1,n-1}
        \]


        \item Multi-Dimensional case: Hotelling's $ T^2 $
        \[
            T^2 =[\sqrt{n}(\bar{X}-\mu_0)'](S^2)^{-1}[\sqrt{n}(\bar{X}-\mu _0)] \sim N_p(0,\Sigma )'\dfrac{W_p(n-1,\Sigma )}{n-1}N_p(0,\Sigma )=\dfrac{p}{n-p}(n-1)F_{p,n-p}
        \]

        And we can get the distribution of \textbf{Hotelling's $ T^2 $}:
         \[ \dfrac{n-p}{p}\dfrac{T^2}{n-1}\sim F_{p,n-p} \]

        Rejection Rule:
        \[
            T^2>\dfrac{p(n-1)}{n-p}F_{p,n-p,\alpha } 
        \]
        
        

        Property:

        Invariant for $ X $ transform: For $ Y=CX+d $, then 
            
            \[
                T^2_Y=n(\bar{X}-\mu_0)'S^{-1}(\bar{X}-\mu_0)=T^2_X 
            \]
    \end{itemize}      
            
\begin{point}
    \hypertarget{PartHotellingT2Test}{LRT of $ \hat{\mu} $}
\end{point}

    Monovariate case see sec.\ref{SubSectionLRT}.

    LRT uses the statistic:
    \[
        \Lambda =\dfrac{\max_{H_0 }L(\mu_0,\Sigma)}{\max_{H_0\cup H_1}L(\mu,\Sigma)}=(1+\dfrac{T^2}{n-1})^{-n/2} 
    \]

    where $ T^2=n(\bar{x}-\mu_0)'S^{-1}(\bar{x}-\mu_0) $
    
\item \textbf{Two-Population Hypothesis Testing}:

    Conduct hypothesis testing to $ \delta =\mu _1-\mu _2 $:
\[
    H_0: \delta =\delta _0\longleftrightarrow H_1:\delta \neq \delta _0
\]

    Notation: The two sample of size $ n_1,n_2 $, each denoted as
    \[
        X_{1,ij}\qquad X_{2,ij} 
    \]
    
    with mean $ \mu_1,\mu_2 $ and covariance matrix $ \Sigma_1,\Sigma_2 $

    \begin{itemize}[topsep=2pt,itemsep=2pt]
        \item Paired Samples: $ n_1=n_2 $
        
        For two paires samples $ \{X_{1,ij}\} $, $ X_{2,ij} $, take subtraction as 
        \[
            D_{ij}=X_{1,ij}-X_{2,ij} 
        \]

        denote $ \bar{D}=\dfrac{1}{n}\sum_{j=1}^nD_{ j} $, $ S^2_D=\dfrac{1}{n-1}\sum_{j=1}^n(D_j-\bar{D}'(D_j-\bar{D})) $
        
        and conduct test to 
        \[
            H_0: \bar{D} =\delta _0\longleftrightarrow H_1:\bar{D} \neq \delta _0
        \]

        And the folloeing steps are as in One-population testing, test
        \[
            T^2=n(\bar{D}-\delta )'(S^2_D)^{-1}(\bar{D}-\delta )\sim \dfrac{(n-1)p}{n-p}F_{p,n-p}
        \]
        
        \item Under Equal Unknown Variance: $ \Sigma_1=\Sigma_2 $
        
        \begin{align}
            \bar{X}_1&=\dfrac{1}{n_1}\sum_{j=1}^{n_1} X_{1,j}&\bar{X}_2&=\dfrac{1}{n_2}\sum_{j=1}^{n_2} X_{1,j}\\
            S^2_1&=\dfrac{1}{n_1-1}\sum_{j=1}^{n_1}(X_{1,j}-\bar{X}_1)(X_{1,j}-\bar{X}_1)'&S^2_2&=\dfrac{1}{n_2-1}\sum_{j=1}^{n_2}(X_{2,j}-\bar{X}_2)(X_{2,j}-\bar{X}_2)'
        \end{align}
        
        And denote pooled variance
        \[
            S^2_\mathrm{pooled}=\dfrac{1}{n_1+n_2-2}\left((n_1-1)S^2_1+(n_2-1)S^2_2 \right)  \sim \dfrac{W_p(n_1+n_2-2,\Sigma )}{n_1+n_2-2}
        \]
        
        Under $ H_0 $, we have 
        \[
            T^2= \dfrac{1}{\frac{1}{n_1}+\frac{1}{n_2}}(\bar{X}_1-\bar{X}_2-\delta _0)'(S^2_\mathrm{pooled})^{-1}(\bar{X}_1-\bar{X}_2-\delta _0)\sim \dfrac{p(n_1+n_2-2)}{n_1+n_2-p-1}F_{p,n_1+n_2-p-1}
        \]
        

    \end{itemize}
    
    


\end{itemize}




    


\subsubsection{Confidence Region}


    Estimate the confidence region for $ \mu $ of $ X\sim N_p(\mu,\Sigma ) $, Monovariate case see sec.\ref{SubSectionConfidenceIntervalForDistributions}
\begin{itemize}
    \item Confidence Region:

    Also use Hotelling's $ T^2 $
    \[
         \dfrac{n-p}{p}\dfrac{T^2}{n-1}\sim F_{p,n-p}
    \]
    
    And take $ 100(1-\alpha )\% $ confidence region of $ \mu $ as
    \[
        R(x)=\{x|T^2\leq c^2\}\qquad c^2=\dfrac{p}{n-p}(n-1)F_{p,n-p,\frac{\alpha }{2}} 
    \]

    The shape of $ R(x) $ is an ellipsoid.
    


    \item Individual Converage Interval
    
    Use the decomposition of $ S^2 $ as a positive finite matrix $ S^2=A^TA $, where $ A $ is some $ p\times p $ matrix, then
    \[
        T^2= [\sqrt{n}(\bar{X}-\mu_0)'](S^2)^{-1}[\sqrt{n}(\bar{X}-\mu _0)]=[A^{-1\prime}\sqrt{n}(\bar{X}-\mu_0)]'[A^{-1\prime}\sqrt{n}(\bar{X}-\mu_0)]
    \]

    Thus denote $ Z=A^{-1\prime}(X-\mu_0) \sim N_p(0,A^{-1\prime}\Sigma A^{-1}) $, the $ T^2 $ estimator of $ Z $ would be
    \[
        T_Z^2=[\sqrt{n}\bar{Z}]'(S_Z^2)^{-1} [\sqrt{n}\bar{Z}]=n\bar{Z}'\bar{Z}=\dfrac{1}{n}\sum_{i=1}^n\bar{Z}_i^2\sim F_{p,n-p}
    \]

    As a simplified case, we can take the \textbf{Individual Converage Interval} of $ Z_i $, which is 
    \[
        \dfrac{\sqrt{n}Z_i}{s_{Z_i}}\sim t_{n-1} 
    \]
    
    And we can take the Confidence Region\footnote{The confidence region of $ Z $ can be transformed to that of $ X $ using $ \hat{Z}=A^{-1\prime}(\hat{X}-\bar{X}) $. } as
    \[
        R(z)= \bigotimes\limits_{i=1}^n(\bar{Z}_i\pm s_{Z_i}t_{n-1,\frac{\beta }{2}})
    \]
    
    where $ \beta  $ take 
    \[
        1-p\beta =1-\alpha  
    \]
    
    Note: Consider that
    \[
        P(\text{all }Z_i\text{ in CI}_i)\geq 1-m\beta =1-\alpha  
    \]

    So the real CR for $ \mu $ should be larger.
    
    
    
    
    
    
    
    


    The shape of $ R(x) $ is an oblique cubold.
    
\end{itemize}
        





\subsubsection{Large Sample Multivariate Inference}
    Basic point:
    \[
        \bar{X}\xrightarrow[]{\mathscr{L}} \mu\qquad S^2\xrightarrow[]{\mathscr{L}} \Sigma  
    \]
\begin{itemize}[topsep=2pt,itemsep=2pt]
    \item One-sample Mean:
    
    \[
        n(\bar{X}-\mu)(S^2)^{-1}(\bar{X}-\mu)\xrightarrow[]{\mathscr{L}} \chi^2_p 
    \]

    \item Unequal Variance Two-sample Mean:
    
    \[
        \bar{X}_1-\bar{X}_2\xrightarrow[]{\mathscr{L}} N(\mu_1-\mu _2,\dfrac{1}{n_1}\Sigma _1+\dfrac{1}{n_2}\Sigma _2) \qquad\quad \dfrac{1}{n_1}S_1^2+\dfrac{1}{n_2}S_2^2\xrightarrow[]{\mathscr{L}} \dfrac{1}{n_1}\Sigma _1+\dfrac{1}{n_2}\Sigma _2
    \]

    Test:
    \[
        T^2=\left[(\bar{X}_1-\bar{X}_2)-(\mu _1-\mu _2) \right]'(\dfrac{1}{n_1}S_1^2+\dfrac{1}{n_2}S_2^2)^{-1}\left[(\bar{X}_1-\bar{X}_2)-(\mu _1-\mu _2) \right]\xrightarrow[]{\mathscr{L}} \chi^2_p
    \]
    
    
    
    
\end{itemize}

    

    
    
    
    















