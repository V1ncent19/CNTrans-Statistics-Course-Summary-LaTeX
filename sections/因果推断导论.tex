\section{因果推断导论部分}\label{SecCausalInference}
\begin{center}
    Instructor: Wanlu Deng
\end{center}



\subsection{Neyman-Rubin Framework}
    Neyman-Rubin Framework\index{Neyman-Rubin Framework} (Donald B.Rubin, 1978), also called Potential Outcome Framework\index{Potential Outcome Framework} is based on \textbf{counter-factual outcome} inference to judge causal effect. 


% \begin{point}
%     Motivation of N-R Model: Difference Between `Correlation' and `Causality'
% \end{point}
    
%         Assume we now has a set of $ \{(W_i,Y_i)\} $ where $ W_i $ happens before $ Y_i $ and $ i $ for $ i^\mathrm{th}  $ object.
    
%         \begin{itemize}[topsep=2pt,itemsep=0pt]
%             \item \textbf{Correlation} describes the relation between $ W_i $ and $ Y_i $;
%             \item while \textbf{Causality} describes the relation between $ Y_i $ and some \textbf{unseen}  $ \tilde{Y}_i $ corresponding to \textbf{What If } $ W_i $ takes another value.
%         \end{itemize}
        
%         Their difference is significant, correlation is mostly based on objective data, while causality contains a lot about how we \textbf{`imagine'} what did not happen, and compare to reality. The $ Y_i-\tilde{Y}_i $ is causal effect (Note that they both has $ _i $, acting on the same object, the different causal effect on different unit also make it not so useful to increase sample size).

\subsubsection{Description of Causal Effect and Challenge}
    Causality concerns `what would happen when \textbf{an action} is applied to \textbf{a unit}'. Here the `unit' is how causality is different from correlation.
\begin{itemize}[topsep=2pt,itemsep=0pt]
    \item A unit is the physical object at that specific time, which is similar to the event in Einstein's relativity.\footnote{Which means that one object at different time ($ (x,t) $ \& $ (x,t') $) is not the same unit (event). However if the assumption of time independency is valid, then object in different time could be the same unit (usually less resonable for human subjects).}
    \item An action is the treatment/intervention that could be \textbf{potentially} applied to the unit. 
\end{itemize}

    In this section we mainly focus on cases with binary intervention, i.e.\footnote{Habitually we denote the more `active' intervention as treatment, but in mathematical form they are symmetric.}
    \begin{align*}
        \{\mathrm{treatment},\,\mathrm{control} \}=\{1,0\} 
    \end{align*}

    To estimate the causal effect (on a unit), we need to obtain both potential outcomes of $ Y(1) $ and $ Y(0) $, but in the real world we can only observe one of them, say, the patient took the medicine, and we got $ Y(1) $, while $ Y(0) $ is missing.

    With this notation, the causal effect could be expressed by the \textbf{estimand} as follows by  comparing the \textbf{potential outcomes}, here's the most commonly used form:
    \begin{align*}
        \tau:=Y_\mathrm{treatment} -Y_\mathrm{control} =Y(1)-Y(0)
    \end{align*}

    In the following parts, we use indicator $ W_i\in\{1,0\} $ to represent the treatment/control \textbf{assigned} to unit $ i $.

\begin{point}
    SUTVA Assumption (Stable Unit Treatment Value Assumption)\index{SUTVA (Stable Unit Treatment Value Assumption)}
\end{point}

    To solve the problem of omitted treatment (e.g. $ Y_i\in\{Y_i(0),Y_i(1),Y_i(2)\} $), and the intervention between units (e.g. $ Y_i(W_{j=1:N}) $) to simplify the model, we usually put the assumption of SUTVA, which has two components:
\begin{itemize}[topsep=2pt,itemsep=0pt]
    \item No Interference
    \begin{align*}
        Y_i(W_{j=1:N})=Y_i(W_i) 
    \end{align*}
    \item No Hidden Variation of Treatment:
    \begin{align*}
        Y_i(W_{j=1:N})=Y_i(W_i)\in\{Y_i(1),Y_i(0)\},\quad W_i\in \{1,0\}:=\mathbb{T}_i=\mathbb{T}
    \end{align*}
\end{itemize}

    In this way the model is simplified, and we could expressed some population/subpopulation level causal estimand.


    \begin{table}[H]
        \centering
        \renewcommand\arraystretch{1}
        \caption{Illustration of Causal Data}
        \begin{tabular}{cccccc}
            \hline
            \hline
            &\multicolumn{2}{c}{Potential Outcomes}&Assignment&Observation&Causal Estimand\\
            \cline{2-3}
            Unit $ i $&$ Y_i(1) $&$ Y_i(0) $&$ W_i $&$ Y^\mathrm{obs}_i  $&$ Y_i(1)-Y_i(0) $\\
            \hline
            \# 1&$ \color{brown}Y_1(1) $&$ \color{gray}Y_1(0) $&$ W_1={\color{brown}1} $&$ Y^\mathrm{obs}_1=\color{brown}Y_1(1)  $&$ {\color{brown}Y_1(1)}-{\color{gray}Y_1(0)} $\\
            \# 2&$ \color{gray}Y_2(1) $&$ \color{brown}Y_2(0) $&$ W_2={\color{brown}0} $&$ Y^\mathrm{obs}_2=\color{brown}Y_2(0)  $&$ {\color{gray}Y_2(1)}-{\color{brown}Y_2(0)} $\\
            \# 3&$ \color{gray}Y_3(1) $&$ \color{brown}Y_3(0) $&$ W_3={\color{brown}0} $&$ Y^\mathrm{obs}_3=\color{brown}Y_3(0)  $&$ {\color{gray}Y_3(1)}-{\color{brown}Y_3(0)} $\\
            \# 4&$ \color{brown}Y_4(1) $&$ \color{gray}Y_4(0) $&$ W_4={\color{brown}1} $&$ Y^\mathrm{obs}_4=\color{brown}Y_4(1)  $&$ {\color{brown}Y_4(1)}-{\color{gray}Y_4(0)} $\\
            $ \vdots $&$ \vdots $&$ \vdots $&$ \vdots $&$ \vdots $&$ \vdots $\\
            \hline
            \hline
        \end{tabular}
        \label{}
    \end{table}
    
     

    

    
\subsubsection{Key Elements and Notations}

\begin{itemize}[topsep=2pt,itemsep=0pt]
    \item \textbf{Unit}: The atomic object in causal inference. $ i=1,2,\ldots,N $
    \item \textbf{Treatment} $ W_i $: (possible) assignment.
    \begin{itemize}[topsep=2pt,itemsep=0pt]
        \item Treatment Group: Set of $ \{\mathrm{Unit}_i|W_i=1\} $;
        \item Controlled Group: Set of $ \{\mathrm{Unit}_i|W_i=0 \} $.
    \end{itemize}
    \item \textbf{Potential Outcome}(PO) $ Y_i $\index{PO (Potential Outcome)}: For each unit with action  treatment(or control), the potential outcome $ Y(W=w),\,w=0,1 $ is the `Eigen Outcome' of the model, despite of what really happens. It can be seen as what would happen when the operation had not been done.
    \item \textbf{Observed Outcome} $ Y_i^\mathrm{obs}  $: The actually happened outcome, $ Y_i^\mathrm{obs}=Y_i(W=w_\mathrm{REAL\_CASE}):=Y_i(W=w_i^\mathrm{obs} ) $.
    
    \item \textbf{Missing Outcome} $ Y_i^{\mathrm{mis} } $: The potential outcome when the $ w-i^\mathrm{mis}:= !w_i^\mathrm{obs}  $ would have been operated (it does exist but we cannot observe the `world-line' where $ w^{\mathrm{mis} }_i $ was operated, thus is unknown to us), $ Y_i^{\mathrm{mis} }=Y_i(W=1-w_\mathrm{REAL\_CASE}):=Y_i(W=w_i^{\mathrm{mis} }) $ 
    \begin{align*}
        Y^\mathrm{obs} _i=Y_i(W_i^\mathrm{obs} )=&\begin{cases}
            Y_i(1)&W_i=1\\
            Y_i(0)&W_i=0
        \end{cases}\\
        Y^{\mathrm{mis} }_i=Y_i(1-W_i^\mathrm{obs} )=&\begin{cases}
            Y_i(0)&W_i=1\\
            Y_i(1)&W_i=0
        \end{cases}
    \end{align*}
    \item \textbf{Causal Effect}\index{Causal Effect} $ \tau_i $ (defined by difference of PO): Difference between potential outcome, $ \tau=Y_i(W=1)-Y_i(W=0)=Y_i(1)-Y_i(0) $
    \item \textbf{Pre-Treatment Variables/Covariates} $ X_i $\index{Pre-Treatment Variable}\index{Covariate}: Some background elements that might attribute to treatment selection/potential outcome. Anyway they may cause confusion to causal inference. For example, the gender of patients $ X_i\in\{\mathrm{female}, \mathrm{male}  \}:=\{1,0\} $.
    \item \textbf{Subgroup}: Treatment/Contorl group could be further divided in subgroup according to covariates.
\end{itemize}


\begin{point}
    \textbf{Causal Effect}  
\end{point}

\begin{itemize}[topsep=2pt,itemsep=0pt]
    \item \textbf{ATE}: Average Treatment Effect (ATE) of the whole population:
    \[
        \mathrm{ATE}=E(Y(1)-Y(0))  
    \]

    Estimand:
    \[
        \hat{\mathrm{ATE} }=\dfrac{1}{N}\sum_{i=1}^N(Y_i(1)-Y_i(0)) 
    \]

    \item \textbf{ATT}/\textbf{ATC}: Average Treatment Effect of Treated/Controlled Group (ATT/ATC):
    \[
        \mathrm{ATT}=E(Y(1)|w=1)-E(Y(0)|w=1)  \qquad \mathrm{ATC}=E(Y(1)|w=0)-E(Y(0)|w=0) 
    \]

    Estimand:
    \[
        \hat{\mathrm{ATT} }=\dfrac{1}{N_t}\sum_{i:w_i=1}(Y_i(1)-Y_i(0))\qquad \hat{\mathrm{ATC} }=\dfrac{1}{N_c}\sum_{i:w_i=0}(Y_i(1)-Y_i(0)) 
    \]
    
    
    \item \textbf{CATE}: Conditional Average Treatment Effect (CATE): `Conditional' for `given covariant $ X $'
    \[
        \mathrm{CATE}_x=E(Y(1)|X=x)-E(Y(0)|X=x)
    \]
    Estimand: (Denote \# $ X_i=x $ as $ N(x) $)
    \[
        \hat{\mathrm{CATE} }=\dfrac{1}{N(x)}\sum_{i:X_i=x}(Y_i(1)-Y_i(0)) 
    \]
    
    
    % \item \textbf{ITE}: Individual Treatment Effect:
    % \[
    %     \mathrm{ITE}_i=Y_i(W=1)-Y_i(W=0)  
    % \]
\end{itemize}

    Where in Estimands, we only know one of $ Y_i(1)/Y_i(0) $ as $ Y^\mathrm{obs}  $, the $ Y^{\mathrm{mis} } $ needs to be \textbf{predicted} (or at least we need to know about the property).

    



% \subsubsection{Assignment Mechanism}
% \begin{itemize}[topsep=2pt,itemsep=0pt]
%     \item  \textbf{Assignment Mechanism}: is a function $ \{X_i,Y_i(0),Y_i(1)\}\to P(W_i) $: The probability that how a unit would have been given treatment/control.

% \[
%     \sum_{\vec{W}\in \{0,1\}^N}P(\vec{W}|X,Y(1),Y(0))=1 
% \]

%     \item \textbf{Finite Population Propensity Score}: The average unit assignment probability for all unit with $ X_i=x $:
%     \[
%         e(x)=\dfrac{1}{N(x)}\sum_{i:X_i=x}P(W_i=1|X,Y(1),Y(0))
%     \]

%     For $ N(x)=0 $, define $ e(x)=0 $
    
    
% \end{itemize}

    

% \subsubsection{Difficulty and Goal of Causal Inference}
% \begin{point}
%     Difficulty:
% \end{point}
% \begin{itemize}[topsep=2pt,itemsep=0pt]
%     \item \textbf{Inadequate Observation Points}: The \textbf{Causal Effect} we want to study is $ \mathrm{ATE}=E(Y(1)-Y(0))  $. Note that each unit has \textbf{two} potential outcomes $ Y(1) $ and $ Y(0) $ for treated/controlled respectively. However we can only observed one of them $ Y^C=Y(W=w_\mathrm{REAL\_CASE} ) $, i.e. at least \textbf{half of potential outcomes are unobserved}, we need to `predict' the missing point using covariants.
%     \item \textbf{Biased Assignment Mechanism}: Covariants may cause biased treatment/control assignment, from which we can never estimate an unbiased causal effect. 
% \end{itemize}

% \begin{point}
%     Goal:
% \end{point}

%     The goal of causal inference is to estimate the treatment effects with $ \{X,W,Y^F\} $ given.


\subsubsection{Assumptions}

\begin{itemize}[topsep=2pt,itemsep=0pt]
    \item \textbf{SUTVA} (Stable Unit Treatment Value Assumption): A usually reasonable assumption to simplify causal model:
    \begin{itemize}[topsep=2pt,itemsep=0pt]
        \item No interference between units;
        \item No hidden variations of treatment.
    \end{itemize}
    \item \textbf{Individualistic Assignment}: Assignment probability of each unit does \textbf{not} depends on the covariants and PO of other units:
    \[
        P_i(W=w|X,Y(1),Y(0))=P_i(W|X_i,Y_i(1),Y_i(0))^w(1-P_i(W|X_i,Y_i(1),Y_i(0)))^{1-w},\,\forall i=1,2,\ldots,N
    \]

    For simplification, denote
    \[
        P_i(W=1|X,Y(1),Y(0))=q(X,Y(1),Y(0)) 
    \]
    
    

    Under individualistic assignment assumption, propensity score is simplified:
    \[
         e(x)=\dfrac{1}{N(x)}\sum_{i:X_i=x}P(W=1|X,Y(1),Y(0))=\dfrac{1}{N(x)}\sum_{i:X_i=x}P(W=1|X_i,Y_i(1),Y_i(0))
    \]
    
    \item \textbf{Probabilistic Assignment}: Probility for both $ W_i=1 $ and $ W_i=0 $ are non-zero (to ensure a reasonable causal model)
    \[
        0<P(W|X,Y(1)Y(0))<1,\,\forall X,Y(1),Y(0) 
    \]
    
    \item \textbf{Unconfounded Assignment}: Assignment mechanism is independent of PO
    \[
        P(W|X,Y(1),Y(0))=P(W|X)
    \]
\end{itemize}

    
\begin{point}
    With above assumption, assignment mechanism and propemnsity score can be simplified:
\end{point}

\begin{align*}
    \text{Assignment Mechanism:}&P(\vec{W}|X,Y(1),Y(0))=\dfrac{1}{Z}\prod_{i=1}^N q(X_i)^W_i(1-q(X_i))^{1-W_i}\\
\end{align*}


    
   





















\subsection{Pearl Framework}
 (Judea Pearl, 1995)

